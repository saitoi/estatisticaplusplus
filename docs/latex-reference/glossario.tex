\newglossaryentry{codigo_fonte}{
    name={Código Fonte},
    description={É um conjunto de palavras e regras escritos de forma organizada contendo instruções de uma linguagem de programação}
}
\newglossaryentry{linguagem_de_programacao}{
    name={Linguagem de Programação},
    description={É uma estrutura contendo regras semânticas e sintáticas expressas através de um código fonte que pode ser transformado em um programa de computador a partir da compilação ou ser interpretado para assim ser executado},
    see=[Veja também]{codigo_fonte}
}
\newglossaryentry{arquivo_executavel}{
    name={Arquivo Executável},
    description={É um arquivo, interpretado pelo sistema operacional como um programa, que contém instruções do processador em questão, geralmente representadas em binário, para a execução de tarefas no computador}
}
\newglossaryentry{codigo_objeto}{
    name={Código Objeto},
    description={Código escrito em linguagem de máquina (binária) ou em uma linguagem intermediária que pode ser interpretada e executada pelo computador}
}
\newglossaryentry{arquivo_objeto}{
    name={Arquivo Objeto},
    description={Arquivo resultante da compilação de um código fonte. Possui vários formatos como o ELF (Executable and Linking Format) do padrão Unix. Além de código objeto, é composto por um cabeçalho com informações de debug, alocação de memória e símbolos (nome de variáveis e de funções). Podem ser ligados à outros arquivos objetos para formar um arquivo executável ou arquivo de biblioteca},
    see=[Veja também]{codigo_fonte,codigo_objeto,arquivo_executavel}
}
\newglossaryentry{assembly}{
    name={Assembly},
    description={Notação legível para o ser humano do código de máquina, isto é, um apelido que representa uma instrução de máquina, mais fácil de ser entendido do que uma sequência binária}
}
\newglossaryentry{bytecode}{
    name={Bytecode},
    description={Código intermediário gerado pelo interpretador e que será executado diretamente em uma máquina virtual}
}
\newglossaryentry{bootstrapping}{
    name={Bootstrapping},
    description={Processo pelo qual o compilador de uma linguagem é escrito na própria linguagem. Geralmente, o compilador inicial é feito em uma linguagem de baixo nível e depois ele é reescrito na própria linguagem em alto nível}
}


\newglossaryentry{padrao_unix}{
    name={Padrão Unix},
    description={abc...}
}

%\newglossaryentry{}{
%    name={},
%    description={},
%    see=[Veja também]{}
%}