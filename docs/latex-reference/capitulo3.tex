\chapter{Capítulo 3}

\section{Bandeiras de Aviso do Compilador}

O GCC compila programas usando o dialeto GNU da linguagem C como base, que incorpora o padrão ANSI C e várias extensões do GNU C. Essas extensões incluem recursos como declarações de variáveis no meio de um bloco de código, expressões com efeitos colaterais e construtores de atributos, entre outros. No entanto, programas válidos escritos em ANSI C podem conflitar com algumas extensões do GNU C.

Para lidar com esses conflitos, o GCC oferece várias opções de linha de comando, conhecidas como flags, que permitem aos desenvolvedores controlar o comportamento do compilador ao compilar o código fonte. A flag "-ansi" é usada para desabilitar as extensões do GNU C que conflitam com o padrão ANSI C. A flag "-pedantic" desabilita todas as extensões do GNU C, não apenas aquelas que conflitam com o padrão ANSI.

No capítulo anterior, usamos a flag -Wall na compilação do nosso programa. Essa flag é uma combinação de várias flags de aviso especializadas que detectam erros comuns de programação. Cada uma das flags contidas em -Wall pode ser usada individualmente. Algumas dessas flags são:

\begin{itemize}
    \item Wcomment: avisa sobre problemas de formatação em comentários, como comentários dentro de comentários.

    \item Wformat: avisa sobre o uso incorreto de formatação em strings em funções como printf e scanf.

    \item Wunused: avisa sobre variáveis que foram declaradas mas não foram usadas no programa.

    \item Wimplicit: avisa sobre funções que foram usadas sem serem declaradas, o que pode acontecer se esquecer de incluir o arquivo de cabeçalho.

    \item Wreturn-type: avisa sobre funções que não retornam nenhum valor mas que não foram declaradas como "void".
\end{itemize}

    O GCC também inclui outras flags de aviso úteis, como:

\begin{itemize}
    \item W: uma flag geral, semelhante a -Wall, que avisa sobre diversos erros comuns.
    
    \item Wconversion: avisa sobre conversões implícitas de tipo, como entre float e integer, que podem causar resultados inesperados.
    
    \item Wshadow: avisa sobre a declaração de variáveis em um escopo em que elas já foram declaradas.
    
    \item Wtraditional: avisa sobre partes do código que seriam interpretadas de forma diferente por um compilador ANSI/ISO e um pré-ANSI.
\end{itemize}

Se você quiser que o programa pare de compilar se houver qualquer aviso das flags que você aplicou, use a flag -Werror.

Usar flags ao compilar um programa é uma boa prática, mas a grande quantidade pode tornar a escolha de quais usar difícil. Com isso em mente, recomendamos que, em geral, as seguintes flags sejam usadas:

\begin{scriptsize}
\estilobash
\begin{lstlisting}
$ gcc -ansi -pedantic -Wall -W
\end{lstlisting}
\end{scriptsize}


\section{Usando o Pré-processador}

% Parou de compilar dnv, daqui a pouco volta a funcionar

Como dito no capítulo inicial, pré-processador é o componente do GCC responsável por processar e manipular o código-fonte antes da compilação começar. Ele lê o código-fonte e realiza um conjunto de operações, incluindo substituição de macros, inclusão de arquivos e compilação condicional. Essas operações são definidas por um conjunto de diretivas, que são comandos especiais que começam com o símbolo "\#".

Uma macro é um pedaço de código ao qual é dado um nome e definida por meio do uso da diretiva \#define. Essa diretiva permite que você atribua um nome a um valor, uma expressão ou até mesmo a um bloco de código, tornando mais fácil e conveniente usar esse nome em vez de repetir o código completo sempre que for necessário. Abaixo está um exemplo de código que utiliza macros:

\begin{scriptsize}
\estiloC
\begin{lstlisting}
#include <stdio.h>

#define NUM 10

int main()
{
    printf("The value of NUM is: %d\n", NUM);
    return 0;
}
\end{lstlisting}
\end{scriptsize}

O pré-processador substitui as ocorrências da macro pelo seu conteúdo correspondente. Assim, quando compilarmos e executarmos o arquivo acima, será exibida a mensagem:

\begin{scriptsize}
\estilobash
\begin{lstlisting}
The value of NUM is: 10
\end{lstlisting}
\end{scriptsize}

Podemos querer executar uma parte do programa somente quando certa macro estiver definida. Para isso, delimitamos tal seção do código com as diretivas \#ifdef e \#endif:

% O CODIGO ABAIXO ESTA DANDO PROBLEMA, NAO TIRE ELE DO COMENTARIO

\begin{comment}
\begin{scriptsize}
\estiloC
\begin{lstlisting}
#include <stdio.h>

int main()
{
#ifdef NUM
    printf("The value of NUM is: %d\n", NUM);
#endif
    return 0;
}
\end{lstlisting}
\end{scriptsize}
\end{comment}

Nesse caso, como a macro NUM não está definida no código fonte, nada será exibido ao executar o programa. No entanto, podemos definir a macro NUM quando compilarmos o programa. Para isso, usamos a opção '-DNAME' para definiremos a macro com nome NAME e, se quisermos designar um valor a ela, escrevemos '-DNAME=valor':

\begin{scriptsize}
\estilobash
\begin{lstlisting}
$ gcc -Wall -DNUM=5 programa.c -o programa
$ ./programa
The value of NUM is: 5
\end{lstlisting}
\end{scriptsize}


\section{Compilando para Debugar (NOTAS POR ENQUANTO)}

Existe uma forma de fazer debugging com a flag -g tal que, se o programa crashar, a parte problemática do código pode ser facilmente encontrada.

Esse capítulo é relativamente grande e usar exemplos vai ser bem importante. Escrever isso em um outro capitulo (chapter 4)