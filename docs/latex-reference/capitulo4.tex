\chapter{Capítulo 4}

\section{Compilando para depuração}

Durante o desenvolvimento de software, a depuração (debugging) desempenha um papel fundamental na identificação e correção eficiente de erros no código. Ao compilar programas em C, existe uma prática recomendada para facilitar a depuração: a utilização da opção -g ao chamar o GCC. Essa opção instrui o compilador a incluir informações de depuração no executável gerado. Essas informações, como símbolos de função, variáveis locais e localizações de linha, permitem uma análise detalhada do código durante a depuração. Por exemplo, ao executar o programa em um depurador, é possível definir pontos de interrupção, inspecionar valores de variáveis e rastrear a execução do código passo a passo.

Aqui está um exemplo de código que podemos usar para ilustrar como compilar programas em C para depuração:

\begin{scriptsize}
\estiloC
\begin{lstlisting}[title=exemplo.c]
int foo (int *p);

int main (void)
{
    int *p = 0;
    /* ponteiro nulo */
    return foo (p);
}

int foo (int *p)
{
    int y = *p;
    return y;
}
\end{lstlisting}
\end{scriptsize}

Neste código, temos duas funções: \texttt{main()} e \texttt{foo()}. A função \texttt{main()} inicializa um ponteiro \texttt{p} com o valor nulo (0) e, em seguida, chama a função \texttt{foo()} passando esse ponteiro como argumento. A função \texttt{foo()} recebe um ponteiro como parâmetro e tenta acessar o valor apontado por ele. No entanto, como p é nulo, essa operação resulta em um erro.

Para compilar esse código com a opção de depuração, você pode usar o seguinte comando:

\begin{scriptsize}
\estilobash
\begin{lstlisting}
$ gcc -Wall -g exemplo.c -o exemplo
\end{lstlisting}
\end{scriptsize}

Ao executá-lo, receberemos uma mensagem de erro, indicando que houve uma violação de segmentação (\textit{segmentation fault}). Essa mensagem é exibida quando ocorre uma tentativa de acessar uma área da memória que não é permitida, como no caso em que o ponteiro \texttt{p} aponta para o valor nulo.

\begin{scriptsize}
\estilobash
\begin{lstlisting}
$ ./exemplo
segmentation fault (core dumped)
\end{lstlisting}
\end{scriptsize}

Vamos falar sobre o arquivo \texttt{core} mencionado na mensagem de erro. Ele é um arquivo de despejo de memória que pode ser gerado quando ocorre uma falha grave em um programa e contém informações sobre o estado da memória no momento da falha, sendo útil para analisar e depurar o problema. Nem todos os sistemas geram automaticamente o \texttt{core} por padrão. Se ele não for gerado, essa funcionalidade pode ser habilitada executando o comando abaixo:

\begin{scriptsize}
\estilobash
\begin{lstlisting}
$ ulimit -c unlimited
\end{lstlisting}
\end{scriptsize}

Esse comando define temporariamente o tamanho máximo do arquivo core como ilimitado, permitindo a geração do \texttt{core} em caso de falha no programa. É importante ressaltar que, se ele não tiver sido gerado durante a primeira execução do programa, \texttt{exemplo} terá que ser executado novamente.


\section{Depurando programas}