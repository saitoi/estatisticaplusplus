% ---
% Arquivo com a formatação do TCC.
% Este arquivo não deve ser modificado.
% ---
\usepackage[utf8]{inputenc}
\usepackage[T1]{fontenc}
\usepackage{amsmath}
\usepackage{amssymb,amsfonts,textcomp}
\usepackage{color}
\usepackage{xcolor}
\usepackage{array}
\usepackage{supertabular}
\usepackage{listings}         % Para as linguagens de programação
\usepackage{lastpage}		  % Usado pela Ficha catalográfica
\usepackage{indentfirst}	  % Indenta o primeiro parágrafo de cada seção.
\usepackage{hhline}
\usepackage{hyperref}
\usepackage[pdftex]{graphicx}

\usepackage[portuguese, ruled]{algorithm2e} % Para algoritmos
\usepackage{algpseudocode}

% retira as mensagens de aviso do pacote Glossaries
\let\printglossary\relax
\let\theglossary\relax
\let\endtheglossary\relax
% coloca Seções em maiúsculas sem negrito no Sumário
\makeatletter
\let\oldcontentsline\contentsline
\def\contentsline#1#2{%
  \expandafter\ifx\csname l@#1\endcsname\l@section
    \expandafter\@firstoftwo
  \else
    \expandafter\@secondoftwo
  \fi
  {%
    \oldcontentsline{#1}{\normalfont\MakeTextUppercase{#2}}%
  }{%
    \oldcontentsline{#1}{#2}%
  }%
}
\makeatother
% ---
% Pacotes glossaries
% ---
\usepackage[subentrycounter,seeautonumberlist,nonumberlist=true]{glossaries}
% para usar o xindy ao invés do makeindex:
%\usepackage[xindy={language=portuguese},subentrycounter,seeautonumberlist,nonumberlist=true]{glossaries}
% ---
% Citações de referências no formato alfabético e negrito
\usepackage[alf, abnt-emphasize=bf]{abntex2cite} 
% Margens definidas em 25 mm para uso como documento em PDF
% Para imprimir use as seguintes margens:
% \usepackage[left=30mm, top=30mm, right=20 mm, bottom=20mm] {geometry}

\usepackage[margin=25 mm]{geometry}
%---
% O arquivo com o nome dos alunos e dos orientadores é lido aqui.
% Atualizar diretamente no arquivo
%---
%%%%%%%%%%%%%%%%%%%%%%%%%%%%%%%%%%%%%%%%%%%%%%%%%%%%%%%%%%%
% Corrige a fonte dos capítulos, seções, resumos, etc.
%%%%%%%%%%%%%%%%%%%%%%%%%%%%%%%%%%%%%%%%%%%%%%%%%%%%%%%%%%%

\renewcommand{\ABNTEXchapterfont}{\bfseries \rmfamily}  % Capítulos em Bold e Maiúsculas
\renewcommand{\ABNTEXchapterfontsize}{\normalsize}
\renewcommand{\ABNTEXsectionfont}{\rmfamily}            %  Seções em  Maiúsculas apenas
\renewcommand{\ABNTEXsectionfontsize}{\normalsize}
\renewcommand{\ABNTEXsubsectionfont}{\bfseries}         % Subseções em Bold apenas
\renewcommand{\ABNTEXsubsectionfontsize}{\normalsize}
\renewcommand{\lstlistingname}{Código}                  % Nome para os códigos no texto
\renewcommand{\lstlistlistingname}{Lista de \lstlistingname s}
\makeatletter                                          % Configura a linha da lista de códigos
\renewcommand\l@lstlisting[2]{{\normalfont\@dottedtocline{1}{1.5em}{2em}{Código~#1}{#2}}}
\makeatother
\usepackage{url16023}  % para retirar < e > da URL nas referências.
%%%%%%%%%%%%%%%%%%%%%%%%%%%%%%%%%%%%%%%%%%%%%%%%%%%%%%%
% Redefine a macro para imprimir a capa 
%%%%%%%%%%%%%%%%%%%%%%%%%%%%%%%%%%%%%%%%%%%%%%%%%%%%%%%
\renewcommand{\imprimircapa}{%
\begin{capa}%
\center
\imprimirinstituicao
\par
\vspace*{1cm}
\MakeUppercase{\imprimirautor}
\vfill
\begin{center}
\imprimirtitulo
\end{center}
\vfill
\imprimirlocal
\par
\imprimirdata
\vspace*{1cm}
\end{capa}
}
%%%%%%%%%%%%%%%%%%%%%%%%%%%%%%%%%%%%%%%%%%%%%%%%%%%%%%%
% Redefine a macro para imprimir a folho de rosto
%%%%%%%%%%%%%%%%%%%%%%%%%%%%%%%%%%%%%%%%%%%%%%%%%%%%%%%
\renewcommand{\imprimirfolhaderosto}{%
\begin{capa}%
\center
\par
\vspace*{1cm}
\MakeUppercase{\imprimirautor}
\vfill
\begin{center}
\imprimirtitulo
\end{center}
\vfill
\hspace*{\fill}\parbox[b]{.5\textwidth}{%
        \linespread{1}\selectfont
\imprimirpreambulo
}
\vfill
\flushright
\vfill
\begin{center}
\large\imprimirlocal
\par
\large\imprimirdata
\vspace*{1cm}
\end{center}
\end{capa}
}
%%%%%%%%%%%%%%%%%%%%%%%%%%%%%%%%%%%%%%%%%%%%%%%%%%%%%%%
% Informações do PDF inseridas automaticamente
% Atualizar apenas as palavras-chave se necessário
% Não modificar as cores dos links e referências
%%%%%%%%%%%%%%%%%%%%%%%%%%%%%%%%%%%%%%%%%%%%%%%%%%%%%%%
\makeatletter
\hypersetup{
pdftitle={\@title},
pdfauthor={\@author},
pdfsubject={\imprimirpreambulo},
pdfcreator={LaTeX with abnTeX2},
colorlinks=true,
linkcolor=black,
citecolor=black,
urlcolor=black
}
\makeatother
%%%%%%%%%%%%%%%%%%%%%%%%%%%%%%%%%%%%%%%%%%%%%%%%%%%%
% Definição das Linguagens de Programação
%%%%%%%%%%%%%%%%%%%%%%%%%%%%%%%%%%%%%%%%%%%%%%%%%%%%
\definecolor{dkgreen}{rgb}{0,0.6,0}
\definecolor{gray}{rgb}{0.5,0.5,0.5}
\definecolor{purple}{rgb}{0.8,0,0.3}
\definecolor{orange}{rgb}{1,0.4,0}
\definecolor{lightlightgray}{rgb}{.95,.95,.95}
\definecolor{lightgray}{rgb}{.9,.9,.9}
\definecolor{lightgray2}{rgb}{.85,.85,.85}
\definecolor{darkgray}{rgb}{.4,.4,.4}
\definecolor{verde}{rgb}{0,0.5,0}
%---
% Comando para inserir a listagem de código, tem 4 parâmetros
% Linguagem, Caption, Label, Nome do Arquivo
%---
\newcommand{\includecode}[4][C]{\mbox{\lstinputlisting[caption=#2, label=#3, escapechar=, style=custom#1]{#4}}}

%---
%Define características em comum e numeração sequencial 
%---
\lstset{numberbychapter=false,framexleftmargin=5mm,  frame=shadowbox, rulesepcolor=\color{gray}}
%---
% A leitura da formatação das linguagens é feita aqui
\definecolor{verde}{rgb}{0,0.5,0}
\newcommand{\estiloC}{
\lstset{
  language=C,
  basicstyle=\ttfamily\small,
  keywordstyle=\color{blue},
  stringstyle=\color{verde},
  commentstyle=\color{red},
  extendedchars=true,
  showspaces=false,
  showstringspaces=false,
  numbers=left,
  numberstyle=\tiny,
  breaklines=true,
  backgroundcolor=\color{green!10},
  breakautoindent=true,
  captionpos=b,
  xleftmargin=0pt,
  extendedchars=false,
  inputencoding=utf8,
  literate={á}{{\'a}}1 {é}{{\'e}}1 {í}{{\'i}}1 {ó}{{\'o}}1 {ú}{{\'u}}1
    {Á}{{\'A}}1 {É}{{\'E}}1 {Í}{{\'I}}1 {Ó}{{\'O}}1 {Ú}{{\'U}}1
    {à}{{\`a}}1 {è}{{\`e}}1 {ì}{{\`i}}1 {ò}{{\`o}}1 {ù}{{\`u}}1
    {À}{{\`A}}1 {È}{{\'E}}1 {Ì}{{\`I}}1 {Ò}{{\`O}}1 {Ù}{{\`U}}1
    {ä}{{\"a}}1 {ë}{{\"e}}1 {ï}{{\"i}}1 {ö}{{\"o}}1 {ü}{{\"u}}1
    {Ä}{{\"A}}1 {Ë}{{\"E}}1 {Ï}{{\"I}}1 {Ö}{{\"O}}1 {Ü}{{\"U}}1
    {â}{{\^a}}1 {ê}{{\^e}}1 {î}{{\^i}}1 {ô}{{\^o}}1 {û}{{\^u}}1
    {Â}{{\^A}}1 {Ê}{{\^E}}1 {Î}{{\^I}}1 {Ô}{{\^O}}1 {Û}{{\^U}}1
    {œ}{{\oe}}1 {Œ}{{\OE}}1 {æ}{{\ae}}1 {Æ}{{\AE}}1 {ß}{{\ss}}1
    {ç}{{\c c}}1 {Ç}{{\c C}}1 {ø}{{\o}}1 {å}{{\r a}}1 {Å}{{\r A}}1
    {€}{{\EUR}}1 {£}{{\pounds}}1
}}

\newcommand{\estilobash}{
\lstset{
    language=bash,
    basicstyle=\ttfamily\small,
    numberstyle=\footnotesize,
    backgroundcolor=\color{gray!10},
    frame=single,
    tabsize=2,
    rulecolor=\color{black!30},
    title=\lstname,
    escapeinside={\%*}{*)},
    breaklines=true,
    breakatwhitespace=true,
    framextopmargin=2pt,
    framexbottommargin=2pt,
    extendedchars=false,
    inputencoding=utf8
}}